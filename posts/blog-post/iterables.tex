% Options for packages loaded elsewhere
\PassOptionsToPackage{unicode}{hyperref}
\PassOptionsToPackage{hyphens}{url}
%
\documentclass[
  ignorenonframetext,
]{beamer}
\usepackage{pgfpages}
\setbeamertemplate{caption}[numbered]
\setbeamertemplate{caption label separator}{: }
\setbeamercolor{caption name}{fg=normal text.fg}
\beamertemplatenavigationsymbolsempty
% Prevent slide breaks in the middle of a paragraph
\widowpenalties 1 10000
\raggedbottom
\setbeamertemplate{part page}{
  \centering
  \begin{beamercolorbox}[sep=16pt,center]{part title}
    \usebeamerfont{part title}\insertpart\par
  \end{beamercolorbox}
}
\setbeamertemplate{section page}{
  \centering
  \begin{beamercolorbox}[sep=12pt,center]{part title}
    \usebeamerfont{section title}\insertsection\par
  \end{beamercolorbox}
}
\setbeamertemplate{subsection page}{
  \centering
  \begin{beamercolorbox}[sep=8pt,center]{part title}
    \usebeamerfont{subsection title}\insertsubsection\par
  \end{beamercolorbox}
}
\AtBeginPart{
  \frame{\partpage}
}
\AtBeginSection{
  \ifbibliography
  \else
    \frame{\sectionpage}
  \fi
}
\AtBeginSubsection{
  \frame{\subsectionpage}
}

\usepackage{amsmath,amssymb}
\usepackage{iftex}
\ifPDFTeX
  \usepackage[T1]{fontenc}
  \usepackage[utf8]{inputenc}
  \usepackage{textcomp} % provide euro and other symbols
\else % if luatex or xetex
  \usepackage{unicode-math}
  \defaultfontfeatures{Scale=MatchLowercase}
  \defaultfontfeatures[\rmfamily]{Ligatures=TeX,Scale=1}
\fi
\usepackage{lmodern}
\ifPDFTeX\else  
    % xetex/luatex font selection
\fi
% Use upquote if available, for straight quotes in verbatim environments
\IfFileExists{upquote.sty}{\usepackage{upquote}}{}
\IfFileExists{microtype.sty}{% use microtype if available
  \usepackage[]{microtype}
  \UseMicrotypeSet[protrusion]{basicmath} % disable protrusion for tt fonts
}{}
\makeatletter
\@ifundefined{KOMAClassName}{% if non-KOMA class
  \IfFileExists{parskip.sty}{%
    \usepackage{parskip}
  }{% else
    \setlength{\parindent}{0pt}
    \setlength{\parskip}{6pt plus 2pt minus 1pt}}
}{% if KOMA class
  \KOMAoptions{parskip=half}}
\makeatother
\usepackage{xcolor}
\newif\ifbibliography
\setlength{\emergencystretch}{3em} % prevent overfull lines
\setcounter{secnumdepth}{-\maxdimen} % remove section numbering


\providecommand{\tightlist}{%
  \setlength{\itemsep}{0pt}\setlength{\parskip}{0pt}}\usepackage{longtable,booktabs,array}
\usepackage{calc} % for calculating minipage widths
\usepackage{caption}
% Make caption package work with longtable
\makeatletter
\def\fnum@table{\tablename~\thetable}
\makeatother
\usepackage{graphicx}
\makeatletter
\def\maxwidth{\ifdim\Gin@nat@width>\linewidth\linewidth\else\Gin@nat@width\fi}
\def\maxheight{\ifdim\Gin@nat@height>\textheight\textheight\else\Gin@nat@height\fi}
\makeatother
% Scale images if necessary, so that they will not overflow the page
% margins by default, and it is still possible to overwrite the defaults
% using explicit options in \includegraphics[width, height, ...]{}
\setkeys{Gin}{width=\maxwidth,height=\maxheight,keepaspectratio}
% Set default figure placement to htbp
\makeatletter
\def\fps@figure{htbp}
\makeatother

\makeatletter
\@ifpackageloaded{caption}{}{\usepackage{caption}}
\AtBeginDocument{%
\ifdefined\contentsname
  \renewcommand*\contentsname{Table of contents}
\else
  \newcommand\contentsname{Table of contents}
\fi
\ifdefined\listfigurename
  \renewcommand*\listfigurename{List of Figures}
\else
  \newcommand\listfigurename{List of Figures}
\fi
\ifdefined\listtablename
  \renewcommand*\listtablename{List of Tables}
\else
  \newcommand\listtablename{List of Tables}
\fi
\ifdefined\figurename
  \renewcommand*\figurename{Figure}
\else
  \newcommand\figurename{Figure}
\fi
\ifdefined\tablename
  \renewcommand*\tablename{Table}
\else
  \newcommand\tablename{Table}
\fi
}
\@ifpackageloaded{float}{}{\usepackage{float}}
\floatstyle{ruled}
\@ifundefined{c@chapter}{\newfloat{codelisting}{h}{lop}}{\newfloat{codelisting}{h}{lop}[chapter]}
\floatname{codelisting}{Listing}
\newcommand*\listoflistings{\listof{codelisting}{List of Listings}}
\makeatother
\makeatletter
\makeatother
\makeatletter
\@ifpackageloaded{caption}{}{\usepackage{caption}}
\@ifpackageloaded{subcaption}{}{\usepackage{subcaption}}
\makeatother
\makeatletter
\@ifpackageloaded{tcolorbox}{}{\usepackage[skins,breakable]{tcolorbox}}
\makeatother
\makeatletter
\@ifundefined{shadecolor}{\definecolor{shadecolor}{HTML}{245ABE}}{}
\makeatother
\makeatletter
\@ifundefined{codebgcolor}{\definecolor{codebgcolor}{HTML}{f8f8f8}}{}
\makeatother
\makeatletter
\ifdefined\Shaded\renewenvironment{Shaded}{\begin{tcolorbox}[breakable, borderline west={3pt}{0pt}{shadecolor}, enhanced, colback={codebgcolor}, frame hidden, boxrule=0pt, sharp corners]}{\end{tcolorbox}}\fi
\makeatother
\ifLuaTeX
  \usepackage{selnolig}  % disable illegal ligatures
\fi
\usepackage{bookmark}

\IfFileExists{xurl.sty}{\usepackage{xurl}}{} % add URL line breaks if available
\urlstyle{same} % disable monospaced font for URLs
\hypersetup{
  pdftitle={Python Iterators, Iterables, Asynchronous},
  pdfauthor={Sunny Hospital},
  hidelinks,
  pdfcreator={LaTeX via pandoc}}

\title{Python Iterators, Iterables, Asynchronous}
\author{Sunny Hospital}
\date{2024-04-10}

\begin{document}
\frame{\titlepage}

\begin{frame}
Summary Note from RealPython
(https://realpython.com/python-iterators-iterables/)
\end{frame}

\begin{frame}[fragile]{Create iterators using the iterator protocol}
\phantomsection\label{create-iterators-using-the-iterator-protocol}
\textbf{What is Iterator in Python}

An iterator is an object that allows you to iterate over a collection of
data and returns one item at a time and keeps track of the current
state.

\textbf{What Is the Python Iterator Protocol}

While there are iterator objects, you can also create iterator for your
custom class by implementing two special methods called iterator
protocol.

\texttt{.\_\_iter\_\_()} to intialize the iterator and return an
interator object (self)

\texttt{.\_\_next\_\_()} to iterate over iterator and will return the
next value and raise an except \texttt{StopIteration} for the end of the
collection.

\textbf{Types of Iterators}

Iterators can be used to perform various tasks: iterating over a
collection of data to

\begin{itemize}
\tightlist
\item
  return each item
\item
  return transformed item
\item
  return newly generated item
\end{itemize}

\textbf{Iterator examples}

\begin{verbatim}
# Iterator example to return its own value
class MyIterator:
    def __init__(self, sequence):
        self._sequence = seequence
        self._index = 0

    def __iter__(self):
        return self 

    def __next__(self):
        if self._index < len(self._sequence):
            item = self._sequence[self._index]
            self._index += 1
            return item
        else 
            raise StopIteeration 

for item in MyIterator([1, 2, 3])
    print(item)
\end{verbatim}

\begin{verbatim}
# Iterator example to return transformed value
class MySquare:

    def __init__(self, sequence):
        self._sequence = sequence
        self._index = 0

    def __iter__(self):
        return self 
    
    def __next__(self):
        if self._index < len(self._sequence):
            square = self._sequence[self._index] ** 2
            self._index += 1
            return square 
        else:
            raise StopIteration
\end{verbatim}

\begin{verbatim}

# Iterator example to return generated value
class FibonacciIterator:
    def __init__(self, stop=10):
        self._stop = stop
        self._index = 0
        self._current = 0
        self._next = 1

    def __iter__(self):
        return self

    def __next__(self):
        if self._index < self._stop:
            self._index += 1
            fib_number = self._current
            self._current, self._next = (
                self._next,
                self._current + self._next,
            )
            return fib_number
        else:
            raise StopIteration
\end{verbatim}

You can create infinite iterator by skipping \texttt{StopIteration}
\end{frame}

\begin{frame}[fragile]{Create Generator iterators}
\phantomsection\label{create-generator-iterators}
\textbf{What is Generator Iterators}

A generator iterator is a function based iterator and must use
\texttt{yield}. It's simpler than class iterator.

Generator iterator expression is similar to the list comprehension but
with parenthesis instead of brackets.

Similar to Iterator, Generator can also return item itself, transformed
item, and new item

\begin{verbatim}
# Generator example to return its own value
def myGeneratorIter(sequence):
    for item in sequence:
        yield(item)

for i in myGeneratorIter([1, 2, 3])
    print(i)

# Generator expression 
(item for item in [1, 2, 3]) # unlike list [], it uses ()

genExpression = (item for item in [1, 2, 3])
for i in genExpression:
    print(i)


# Generator example to return transformed value
def SquareGenerator(sequence):
    for item in sequence:
        yield(item**2)

for i in SquareGenerator([1,2,3])
    print(i)
    
\end{verbatim}
\end{frame}

\begin{frame}[fragile]{Memory efficient data processing}
\phantomsection\label{memory-efficient-data-processing}
\textbf{Benefits}

\begin{itemize}
\tightlist
\item
  You don't need to store all the data in the computer memory at the
  same time.
\item
  It can decouple processing with data
\item
  Iterators are the only way to process infinite data streams
\end{itemize}

Regular functions or comprehensions for data processing create data
structure such as a list and it stores data in memory at the same time.

Iterators keep only one item in memory at a time, generating the next
ones on demand or lazily.

\textbf{Constraints} * You can't iterate over an iterator. Once
StopIteration is raised, the iterator is exhausted. * You can only move
forward, not backyard. You only have \texttt{\_\_next\_\_()}, not
previous. * unlike lists and tuples, iterators don't allow indexing and
slicing operations with the {[}{]} operator:

\begin{verbatim}
def square_list(sequence):
    squares = []
    for item in sequence:
        squares.append(item**2)
    return squares 
\end{verbatim}

\begin{block}{Creating pipeline with generator iterator}
\phantomsection\label{creating-pipeline-with-generator-iterator}
\begin{verbatim}
def to_square(numbers):
    return (number**2 for number in numbers)

def to_cube(numbers):
    return (number**3 for number in numbers)

def to_even(numbers):
    return (number for number in numbers if number % 2 == 0)

def to_odd(numbers):
    return (number for number in numbers if number % 2 != 0)

def to_string(numbers):
    return (str(number) for number in numbers)

>>> import math_pipeline as mpl

>>> list(mpl.to_string(mpl.to_square(mpl.to_even(range(20)))))
['0', '4', '16', '36', '64', '100', '144', '196', '256', '324']

>>> list(mpl.to_string(mpl.to_cube(mpl.to_odd(range(20)))))
['1', '27', '125', '343', '729', '1331', '2197', '3375', '4913', '6859']
\end{verbatim}
\end{block}
\end{frame}



\end{document}
